\documentclass{article}
\usepackage[utf8]{inputenc}
\usepackage[spanish]{babel}
\usepackage{graphicx}
\usepackage{color}
\graphicspath{ {c:/user/images/} }

\begin{document}
\begin{center}
\includegraphics[scale=0.12]{escudo.png}
\end{center}
\vspace{50pt}
\begin{center}
\bf{\sc\Large 'El origen de la computación'}\\
\end{center}
\vspace{50pt}
\begin{center}
\begin{center}
\bf{\sc\large Por:}\\
\end{center}
\bf{\sc\large María Salomé Garcés Montero}\\
\end{center}
\vspace{50pt}
\begin{center}
\bf{\sc\large Facultad de ingeniería}\\
\end{center}
\begin{center}
\bf{\sc\large Medellín}
\end{center}
\begin{center}
\bf{\sc\large 2020}\\
\end{center}\
\newpage
\begin{center}
\bf{\sc\LARGE El origen de la computación}\\
\end{center}
\vspace{50pt}
\large 
¿Qué es la computación? o ¿De dónde y por qué surge?
No son preguntas que nos hagamos a menudo, quizá preguntas que nunca nos haríamos; porque cuestionarse el porqué o el de dónde de las cosas no es un hábito que muchos adquieran. En este caso, puede ser también porque ignoramos la importancia de esta ciencia en nuestras actividades diarias.
Lo cierto es que estas preguntas que no nos hemos planteado nos pueden llevar a responder otras que seguramente en algún momento se nos han pasado por la cabeza, como por ejemplo cómo funcionan muchas de las cosas a nuestro al rededor: nuestro teléfono celular,laptop,incluso nuestro auto. Así entonces, para dar solución a esto, debemos remontarnos al incio, a lo que originó este mundo conocido como la computación. 

\vspace{10PT}

Todo empezó cuando Greorg Cantor, matemático ruso, basado en los estudios previos de el también matemático Bernard Bolzano. Expuso su llamada teoría de conjuntos, que explicada de forma sencilla propone que todos los conjuntos de números infinitos poseen diferente cantidad de elementos, es decir, existen infinitos más grandes que otros. Pero su propuesta empezó a mostrar falencias cuando se aplicaba en el conjunto de números decimales; lo que desató una crisis entre los matemáticos. Pues habían quienes afirmaban que todos los problemas podían ser resueltos mediante esta área; otros por el contario empezaron a entender sus limitaciones.

¿Quién tendría la razón?

\vspace{10PT}

A inicios del siglo XX David Hilbert quien era uno de los que afirmaba que la matemática era infalible, propuso crear una forma mecánica para determinar la solución a los problemas de razonamiento matemático, un método que pudiese ser aplicado a cualquier enunciado lógico de manera que se determine si este es verdadero o no. Pero su idea fue destruida en 1931 por el austriaco Kurt Gödel quien probó que existen afirmaciones matemáticas cuya veracidad es imposible de determinar mediante un secuencia de pasos mecánicos -lo que hoy se conoce como un algoritmo-.

\vspace{10PT}

En 1931 el joven Alan Turing, basándose en las aseveraciones de Gödel, propuso el teorema de la parada, con el cual se puede establecer si un problema puede o no ser solucionado mediante un numero finito de pasos, como lo proponía Hilbert. Turing llegó a la conclusión de que no todos los problemas pueden ser resueltos mediante tratamientos matemáticos. 
A partir de estas afirmaciones surge la idea de un modelo matemático con el cual se pueda mostrar cuáles problemas tenían solución, este modelo recibió el nombre de 'Máquina de Turing'.

\vspace{10PT}

La Máquina de Turing consistía en una cinta infinita sobre la que se podían escribir caracteres; según el símbolo recibido y un conjunto de intrucciones, se toma la decisión de mover la cinta a la derecha o la izquierda, o de reemplazar el caracter,etc. Convirtiendo así la solución del problema en un conjunto de pasos finitos. A los problemas que se podían resolver por medio de este algoritmo se les denominó 'computables'. 

\vspace{10PT}

Esta máquina marcó el inició de la computación y es el principio que rige el funcionamiento de los dispositivos que hoy usamos comunmente. Todos los procesos que llevamos a cabo en la actualidad, desde el más básico hasta el más complejo involucran un computador. 

\vspace{10PT}

Así pues que cuestionarse ha servido para mucho más que saber el cómo y porqué de las cosas. En el caso de Gödel o Turing, sirvió para dar paso a una de las ciencias más importantes e imprescindibles de la vida moderna.

\newpage
\begin{thebibliography}{0}
  \bibitem{Sautoy2018} Sautoy, M. (2018). Georg Cantor, el matemático que descubrió que hay muchos infinitos y no todos son del mismo tamaño. BBC. 
  
  Recuperado de 
  \href{\textcolor{blue}{https://www.bbc.com/mundo/noticias-45300219}}
  \bibitem{Gonzales2019}  González,E. y Pizarro, A. (2019). Gödel y los límites de las matemáticas. El País.
  
  Recuperado de 
  \href{\textcolor{blue}{https://elpais.com/elpais/2019/01/24/ciencia/1548329597_971134.html}}
  
  \bibitem{Lopez2018} López, P. (2018). La filosofía de las computadoras. La Voz.  
  
  Recuperado de 
  \href{\textcolor{blue}{https://www.lavoz.com.ar/numero-cero/la-filosofia-de-las-computadoras}}
  
   \bibitem{NN2018} ¿Qué aportó a la ciencia Alan Turing? (2018). La Vanguardia.
   
   Recuperado de 
   \href{\textcolor{blue}{https://www.lavanguardia.com/historiayvida/historia-contemporanea/20180611/47312986353/que-aporto-a-la-ciencia-alan-turing.html}}
   
   \bibitem{Ruiza2004} Ruiza, M., Fernández, T. y Tamaro, E. (2004). Biografia de Georg Cantor. Biografías y Vidas. Enciclopedia biográfica en línea.
   
   Recuperado de 
   \href{\textcolor{blue}{https://www.biografiasyvidas.com/biografia/c/cantor.html}}
   
  
\end{thebibliography}
\end{document}
